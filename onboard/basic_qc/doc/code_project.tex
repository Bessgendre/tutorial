\documentclass[parskip=full]{article}
% \usepackage[utf8]{inputenc}
% \usepackage[default]{lato}
\DeclareMathAlphabet{\mathcal}{OMS}{cmsy}{m}{n} % restore mathcal in mathptmx
\usepackage[T1]{fontenc}
\usepackage{algorithm}
\usepackage{algpseudocode}
\usepackage{bm}
\usepackage{amsmath}
\usepackage{amssymb}
\usepackage{cases}
% \usepackage{times}
% \usepackage{mathptmx}
\usepackage{xcolor}
\usepackage{graphicx}
\usepackage{hyperref}
\hypersetup{
    colorlinks=true,
    linkcolor=blue,
    % filecolor=magenta,
    urlcolor=blue,
}
\usepackage{cleveref}
\usepackage{braket}
\usepackage[font={small}]{caption}
\usepackage{mhchem}
\usepackage{authblk}

\usepackage{listings}

\usepackage{parskip}

\usepackage{geometry}
\geometry{
    left=25mm,
    right=25mm,
    top=20mm,
    bottom=25mm,
}

% more vertical space in table
\usepackage{booktabs}
\usepackage[para,online,flushleft]{threeparttable}
% \renewcommand{\arraystretch}{1.5}

% for feynman diagrams
\usepackage{feynmp}
\DeclareGraphicsRule{*}{mps}{*}{}

\usepackage{tikz-feynman}
\tikzfeynmanset{compat=1.1.0}


\usepackage{stackengine}
\newcommand{\vbar}[1]{\stackon[1pt]{$#1$}{\shortmid}}

\newcommand{\mr}[1]{\mathrm{#1}}
\newcommand{\tr}[1]{\textrm{#1}}
\newcommand{\md}{\mathrm{d}}
\newcommand{\mi}{\mathrm{i}}
\newcommand{\me}{\mathrm{e}}
\newcommand{\mat}[1]{\mathbf{#1}}
\newcommand{\mtt}[1]{\mathtt{#1}}
\newcommand{\psum}[2]{\sum_{#1}^{#2}{}^{'}}
\newcommand{\vac}{\ket{\textrm{vac}}}

\definecolor{myred}{rgb}{0.8,0,0}
\newcommand{\hzy}[1]{\textbf{\textcolor{myred}{HZY: #1}}}

% \newcommand{\normalordered}[1]{(#1)_{\mr{N}}}
\newcommand{\normalordered}[1]{\left\{#1\right\}}

\newcommand{\parasep}{\vspace{0.5em}\begin{center}*\hspace{6em}*\hspace{6em}*\end{center}\vspace{0.5em}}

\title{Programming Projects on Basic Quantum Chemistry Methods}
\author[1]{Hong-Zhou Ye\thanks{\href{mailto:hzyechem@gmail.com}{hzyechem@gmail.com}}}
\affil[1]{\normalsize\textit{Department of Chemistry and Biochemistry and Institute for Physical Science and Technology \protect\\ University of Maryland, College Park, MD, 20742}}
\date{Aug 2024}
% \date{}

% \setcounter{section}{-1}

\begin{document}

\maketitle


    This document collects a few programming projects that help you build basic skills for quantum chemistry programming by turning the methods you learned in Szabo \& Ostlund into computer programs.
    There are excellent notes/tutorials available on similar topics:
    \begin{itemize}
        \item Notes by Prof.~David Sherrill: \href{http://vergil.chemistry.gatech.edu/notes/index.html}{link}
        \item Programming Projects by Prof.~Eugene DePrince: \href{https://www.chem.fsu.edu/~deprince/programming_projects}{link}
        \item Programming Projects by Prof.~Daniel Crawford: \href{https://github.com/CrawfordGroup/ProgrammingProjects}{link}
        \item Programming Tutorials by ajz34: \href{https://pycrawfordprogproj.readthedocs.io/en/latest}{link}
    \end{itemize}
    When having a hard time with your own code, try to read relevant materials from these notes/tutorials.


    \section{Hartree-Fock (HF)}

    \subsection{Simple fixed point algorithm}

    Using the AO basis integrals provided in the directory \texttt{reference} for a water molecule in the STO-3G basis, write a function that does the spin-restricted HF (RHF) calculations as follows:
    \begin{verbatim} e_tot, mo_energy, mo_coeff = do_rhf(h, V, S, nocc, enuc)\end{verbatim}
    where
    \begin{itemize}
        \item $h$ and $V$ are the one-electron and two-electron integrals in the AO basis, where $V$ is in the chemists' notations, i.e.,~$(11|22)$.
        $S$ is the AO basis overlap matrix.
        These integrals are stored in \texttt{npy} format and can be loaded using \texttt{numpy.load}.
        \item \texttt{nocc} is the number of occupied orbitals (i.e., half the number of electrons).
        \item \texttt{enuc} is the nuclear repulsion energy, which is  $9.189222125371913$~Ha for the reference water molecule.
    \end{itemize}

    The reference HF total energy from PySCF is $-74.9631463874825$~Ha.

    \subsection{DIIS}

    You may notice that your SCF code, which is based on a simple fixed-point algorithm, takes a few tens of cycles to converge.
    The convergence can be significantly sped up using the DIIS algorith, which stands for \textit{direct inversion of iterative subspace}.
    Read any of the tutorials on DIIS (e.g.,~DePrince \texttt{diis} folder; Crawford \texttt{Project \#8}) and implement DIIS for your SCF code.



    \section{Second-order Møller-Plesset perturbation theory (MP2)}

    MP2 is the simplest electron correlation theory.
    Implementing a function for calculating the MP2 correlation energy using the RHF solutions generated by your SCF code.
    Test your implementation using the water/STO-3G system.
    The reference MP2 energy from PySCF is $-0.0356082692564318$~Ha.

    You may find Crawford \texttt{Project \#4} helpful.



    % \section{MP2 with density fitting}

    % Computing and evaluating the two-electron integrals (or ERIs) both require $O(N^4)$ cost, which soon becomes a computational bottleneck.
    % Density fitting is a commonly invoked approximation to more efficiently handle the ERIs.
    % Read the notes by Prof.~David Sherrill on density fitting (\href{http://vergil.chemistry.gatech.edu/notes/df.pdf}{link}).



\end{document}
